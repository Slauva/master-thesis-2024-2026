\section{Электроэнцефалография и нейрофизиологические основы визуального воображения}

\subsection{Биофизическая природа сигнала электроэнцефалографии}

Электроэнцефалография (electroencephalography, EEG) представляет собой неинвазивный метод регистрации электрической активности мозга с поверхности кожи головы. Измеряемый сигнал отражает суммарные внеклеточные потенциалы, возникающие вследствие трансмембранных токов в популяциях нейронов коры головного мозга.

С биофизической точки зрения основным вкладом в формирование регистрируемого EEG-сигнала являются синхронные постсинаптические токи в пирамидальных нейронах коры, ориентированных перпендикулярно поверхности коры. Благодаря их упорядоченной геометрии и одновременной активации формируются макроскопические электрические диполи, поля которых суммируются и становятся измеримыми на поверхности головы~\cite{10.1038/nrn3241}. Вклад отдельных спайков нейронов в скальповый EEG относительно невелик по сравнению с суммарным вкладом синаптической активности, что подчёркивает коллективную природу регистрируемого сигнала.

Таким образом, EEG отражает не активность отдельного нейрона, а согласованную активность больших нейронных ансамблей. Это обстоятельство имеет принципиальное значение для интерпретации данных: изменения амплитуды или спектра EEG следует рассматривать как индикатор изменений в координации и синхронизации нейронных популяций, а не как прямое отображение локальной клеточной активности.

% ---- КАРТИНКА 1 ----
% Вставить схему:
% "Суммирование диполей пирамидальных нейронов и формирование скальпового потенциала"
% (пирамидальные нейроны, ориентированные перпендикулярно поверхности коры,
% суммирование диполей, распространение поля через череп к электродам)

\subsection{Пространственные ограничения и обратная задача}

Несмотря на высокую временную разрешающую способность (порядка миллисекунд), EEG характеризуется относительно низким пространственным разрешением. Это связано с эффектом объёмной проводимости (volume conduction): электрические поля, генерируемые в коре, распространяются через мозговую ткань, ликвор, кости черепа и кожу головы, что приводит к пространственному «размазыванию» потенциалов~\cite{10.1002/hbm.20745}. В результате один и тот же источник может влиять на множество электродов, а измеренные сигналы на разных каналах оказываются линейными комбинациями вкладов нескольких источников.

Математически задача восстановления распределения источников по измеренным скальповым потенциалам формулируется как обратная задача (inverse problem). Данная задача является некорректной в смысле Адамара: множество различных конфигураций источников может приводить к практически идентичной картине потенциалов на поверхности головы~\cite{10.1186/1743-0003-5-25}. Для получения единственного решения необходимо вводить дополнительные ограничения (регуляризацию), априорные предположения о числе, глубине и распределении источников.

Следствием указанных ограничений является необходимость осторожной интерпретации пространственных различий амплитуд EEG-сигнала. Электрод не соответствует напрямую конкретной анатомической области мозга, а регистрирует интегральный вклад распределённых источников.

% ---- КАРТИНКА 2 ----
% Вставить схему:
% "Иллюстрация обратной задачи EEG"
% (несколько возможных конфигураций источников -> одинаковое распределение потенциалов на скальпе)

\subsection{Электроэнцефалография как пространственно-временной метод анализа}

Традиционный анализ EEG часто основывается на изучении сигналов отдельных электродов или латентностей вызванных потенциалов. Однако более корректным является рассмотрение EEG как пространственно-временного поля потенциалов на поверхности головы~\cite{10.1016/j.neuroimage.2011.12.039}. Анализ топографических карт, глобальной мощности поля и динамики распределения потенциалов позволяет более полно учитывать структуру сигнала.

Такой подход особенно важен в задачах когнитивной нейронауки, где исследуются распределённые нейронные сети. Пространственно-временной анализ снижает риск переинтерпретации локальных максимумов и позволяет выявлять устойчивые топографические паттерны активности.

\subsection{Осцилляторная организация мозговой активности}

EEG-сигнал традиционно анализируется в частотной области, где выделяют основные диапазоны: дельта (0.5–4 Гц), тета (4–8 Гц), альфа (8–13 Гц), бета (13–30 Гц) и гамма (>30 Гц). Осцилляторная активность рассматривается как механизм координации распределённых нейронных ансамблей и межрегиональной коммуникации~\cite{10.1016/S0167-8760-99-00047-1}.

Особое значение для когнитивных процессов имеет альфа-диапазон. Показано, что изменения мощности альфа-ритма связаны с механизмами внимания и контролируемого доступа к хранящейся в памяти информации~\cite{10.1016/j.tics.2012.10.007}. Альфа-осцилляции рассматриваются как механизм функционального подавления нерелевантной информации и временной координации обработки.

Для задач визуального воображения (visual imagery) данный диапазон представляет особый интерес, поскольку воспоминание и внутреннее воспроизведение образа требуют активации хранимых представлений и их селективного извлечения. Таким образом, анализ спектральных характеристик, в частности альфа- и бета-диапазонов, является обоснованным при исследовании реконструкции визуальных стимулов из EEG.

\begin{figure}[!t]
  \centering
  \includegraphics[width=150mm]{freq-bands-eeg.pdf}
  \caption{Частотные диапазоны EEG и их функциональные ассоциации}
  \label{freq-bands-eeg}
\end{figure}

\subsection*{Выводы по разделу}

Электроэнцефалография регистрирует суммарную синхронную активность нейронных популяций, преимущественно отражающую постсинаптические токи в коре головного мозга. Несмотря на высокую временную разрешающую способность, метод характеризуется ограниченным пространственным разрешением вследствие объёмной проводимости и некорректности обратной задачи. Осцилляторная структура EEG, особенно в альфа-диапазоне, тесно связана с процессами внимания и памяти, что создаёт теоретические основания для анализа визуального воображения и реконструкции образов на основе спектральных и пространственных характеристик сигнала.