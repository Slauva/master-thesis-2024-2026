\newpage
\begin{center}
  \textbf{\large ВВЕДЕНИЕ}
\end{center}
\addcontentsline{toc}{chapter}{ВВЕДЕНИЕ}


\textbf{Актуальность}

В последние десятилетия наблюдается интенсивное развитие методов декодирования мозговой активности и построения интерфейсов мозг-компьютер (brain-computer interface, BCI),
основанных на анализе сигналов электроэнцефалографии (electroencephalography, EEG). 
Применение методов машинного обучения (machine learning, ML) и глубокого обучения (deep learning, DL) 
существенно повысило качество классификации когнитивных состояний и сенсорных стимулов по данным EEG\cite{10.1088/1741-2552/aab2f2, 10.1016/j.jneumeth.2019.108378}. 

Одним из перспективных направлений является реконструкция визуальных стимулов на основе мозговой активности. 
Большинство существующих исследований сосредоточено на задачах визуального восприятия (visual perception), при которых испытуемому предъявляется внешний стимул, вызывающий относительно стабильный сенсорный ответ. 
В подобных парадигмах достигаются более высокие показатели декодирования по сравнению с задачами, 
основанными на визуальном воображении (visual imagery)\cite{10.3390/electronics11172706,10.1109/TNSRE.2024.3410870,10.1016/j.eswa.2025.127076}.

В то же время задачи реконструкции или классификации образов, возникающих в состоянии воображения или воспоминания, характеризуются существенно большей сложностью. 
Показано, что временная динамика нейронных представлений при imagery отличается от perception, что указывает на различие механизмов обработки информации\cite{10.7554/eLife.33904}. 
Кроме того, общность представлений между perception и imagery проявляется преимущественно в ограниченных частотных диапазонах, в частности в альфа-диапазоне (alpha band), 
что свидетельствует о специфике спектральной организации воображаемых образов\cite{10.1016/j.cub.2020.04.074}. 

Экспериментальные исследования демонстрируют, что точность классификации imagery-сигналов, как правило, 
ниже по сравнению с perception, особенно в многоклассовых задачах и при межсубъектной проверке\cite{10.3390/electronics11172706,10.1109/TNSRE.2024.3410870,10.1016/j.eswa.2025.127076}. 
Это указывает на меньшую устойчивость и более выраженную вариабельность нейронных паттернов в фазе воображения.

В области реконструкции изображений из мозговой активности значительный прогресс достигнут при использовании функциональной магнитно-резонансной томографии (functional magnetic resonance imaging, fMRI), 
обладающей высоким пространственным разрешением. 
Современные работы в направлении EEG-to-Output активно применяют генеративные модели, 
включая генеративные состязательные сети (Generative Adversarial Networks, GAN) и диффузионные модели (diffusion models)\cite{10.48550/arxiv.2412.19999}. 
Однако большинство таких исследований ориентировано на реконструкцию естественных изображений и использует сложные семантические латентные пространства, 
что затрудняет интерпретацию и требует больших объёмов данных.

Локально-ориентированная реконструкция простых бинарных изображений малого разрешения непосредственно из EEG, 
зарегистрированной в фазе воспоминания, остается ограниченно исследованной. 
Дополнительную сложность создают низкое пространственное разрешение EEG, высокая шумность сигнала, межсубъектная вариабельность и временная нестационарность нейронной активности. 

Таким образом, актуальной научной задачей является исследование возможности реконструкции простых бинарных визуальных стимулов малого разрешения по многоканальным сигналам EEG, 
зарегистрированным в фазе воспоминания, с использованием методов ML и анализа пространственно-спектральных характеристик сигнала. 
Такая постановка позволяет сосредоточиться на фундаментальном вопросе: содержит ли EEG в фазе воспоминания достаточную информацию для восстановления структуры зрительного образа на уровне отдельных пикселей.


\newpage

\textbf{Цель магистерской квалификационной работы} — разработать и исследовать методы реконструкции бинарных визуальных стимулов размером $6\times6$ пикселей на основе многоканальных сигналов EEG, зарегистрированных в фазе воспоминания ранее предъявленных изображений, и установить связь между пространственно-спектральными характеристиками EEG и качеством реконструкции изображения.

\textbf{Задачи магистерской квалификационной работы:}
\begin{enumerate}
\item Провести анализ современных исследований в области декодирования и реконструкции визуальных стимулов из EEG, с особым вниманием к различиям между парадигмами perception и imagery.
\item Сформулировать математическую постановку задачи реконструкции бинарного изображения размером $6\times6$ по данным многоканальной EEG.
\item Разработать процедуру предобработки EEG-сигналов, включая фильтрацию, удаление артефактов и извлечение пространственно-спектральных признаков.
\item Реализовать и сравнить несколько моделей ML для реконструкции изображения.
\item Оценить вклад различных частотных диапазонов в задачу реконструкции.
\item Провести количественный анализ качества реконструкции с использованием метрик pixel accuracy, Hamming distance и Intersection over Union (IoU), а также исследовать устойчивость результатов к межсубъектной вариабельности.
\end{enumerate}


\textbf{Научной новизной обладают следующие результаты магистерской квалификационной работы:}
\begin{enumerate}
\item Предложена формализованная постановка задачи локально-ориентированной реконструкции бинарных изображений малого разрешения ($6\times6$) по данным EEG, зарегистрированным в фазе воспоминания, без использования внешних семантических латентных пространств.
\item Установлена зависимость качества реконструкции от пространственно-спектральных характеристик EEG-сигнала, включая вклад отдельных частотных диапазонов.
\item Показано, что использование структурированной постановки задачи и агрегирования информации во временной области повышает устойчивость реконструкции по сравнению с независимой пиксельной классификацией.
\end{enumerate}


\textbf{Публикации:}
\begin{enumerate}
\item Koshman, V. V., Kirilin, A. D., Skvortsova, V. A. (2024). Development of a System for Monitoring Medical Indicators Using Electromyography and Electrocardiography to Calculate Exoskeleton Efficiency. \textit{Russian Journal of Nonlinear Dynamics}, 20(5), 859–874. DOI:10.20537/nd241214.
\item Kirilin, A. D., Skvortsova, V. A., Koshman, V. V. (2024). Development of a Lever-Based Twisted String Actuator for Exoskeleton Systems. \textit{Russian Journal of Nonlinear Dynamics}, 20(5), 827–844. DOI:10.20537/nd241212.
\item Nasybullin, A. A., Abdullaev, N., Baranov, M. A., Koshman, V. V., Mahonin, V. A. (2024). A Methodology to Rank Importance of Frequencies and Channels in Electromyography Data with Decision Tree Classifiers. \textit{Russian Journal of Nonlinear Dynamics}, 20(5), 895–906. DOI:10.20537/nd241216.
\end{enumerate}
